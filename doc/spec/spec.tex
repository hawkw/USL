\input assets/420pre
\usepackage{minted}
\begin{document}
	\MYTITLE{Useless Stack Language 1.0 Specification}
	\MYHEADERS{}
	\tableofcontents
	\vfill
	\pagebreak
	\section{Introduction}
	\section{Grammar}
	\begin{listing}
		\centering
		\begin{grammar}
			<program> ::= <data>
				\alt <data> <program>

			<data> ::= <number> \alt <string> \alt <ident> \alt <obj>

			<obj> ::= `\{'  <program> `\}'
		\end{grammar}
		\caption{Backus--Naur Form for the USL grammar}
	\end{listing}
	USL features only 3 primitive types: numbers, identifiers, and Objects (where an Object is a stack containing more primitive types). Strings, as given by the grammar spec, are syntactic sugar for an object filled with single character identifiers.
	\section{Execution Model}
	\subsection{Data Structures}
	USL’s execution model involves three structures: the Dictionary, Stack A, and Stack B. Stack A and Stack B are both stacks of primitives, to be used during program execution. The Dictionary is a mapping of identifiers $\into$ a stack of primitives, such that identifiers may be bound to other symbols in the language, or to Objects, to serve as compound structures. The binding is a stack such that multiple definitions may be scoped and observed; original meanings of identifiers are restored once an accompanying undefine is executed.

	\subsection{Execution Flow}
		In preparing for program execution, USL first empties its environment. The Dictionary is set to an empty mapping, and Stack A and Stack B are both emptied. Instructions to execute are loaded into Stack A before execution begins.

		During the course of execution, the following process is observed:
		\begin{enumerate}
			\item A primitive is popped off of Stack A. If that primitive is a Number or an Object, it is passed directly to Stack B, otherwise if that primitive is an identifier, that identifier is evaluated.
			\item When an identifier is evaluated, if it has a special implementation meaning, that meaning is then executed. Otherwise, if that identifier both does not have a special meaning and is not given in the Dictionary, it is passed to Stack B. Else, in the case that the identifier can be found in the Dictionary, a Dictionary Evaluation occurs.
			\item A Dictionary Evaluation happens as the following:
			\begin{itemize}
				\item First, look up the identifier in the dictionary.
				\item If the result is a Number, place it on Stack B.
				\item If the result is another Identifier, evaluate it as per above, unless that Identifier is itself, in which case place that identifier on Stack B.
				\item If the result is an object, explode the object.
			\end{itemize}
			\item When exploding an object, one reads out its contents in reverse order and pushes each one in turn onto Stack A.
		\end{enumerate}
	\section{Language Keywords}
	The following identifiers are known to have special meaning in USL:
	\begin{description}
		\item[macro] Pop and hold an Object from Stack B. Rotate Stack A with Stack B, and then explode.
		\item[+] Pop two Numbers from Stack B. Push the result of adding those Numbers back to Stack B.
		\item[-] Pop two Numbers from Stack B. Push the result of subtracting the second by the first back to Stack B
		\item[*] Pop two numbers from Stack B. Push the result of multiplying the numbers.
		\item[/] Pop two numbers from Stack B. Push the result of dividing the second by the first back to Stack B.
		\item[def] Pop an Identifier and then any primitive from Stack B. Push the primitive as a binding to the Identifier.
		\item[undef] Pop an Identifier from Stack B. Pop a binding from that Identifier in the Dictionary. If the binding does not exist, or is empty, destroy the entry from the Dictionary entirely.
		\item[\$] Pop a primitive from Stack B, and push this primitive onto Stack A.
		\item[yank] Pop an Object from Stack B. Pop any primitive from the Object. Then, push the Object back to Stack B, and then the primitive which was popped.
		\item[smush] Pop a primitive from Stack B, and then an Object. Push the primitive onto the Object, and then push the Object back to Stack B.
		\item[\#] Pop a primitive from Stack B. If that primitive is an Identifier with a Dictionary entry, push back the result of that Identifier’s binding. Otherwise, if the primitive was an Object, assemble a new Identifier by gluing all of that Object’s contained identifiers together, and push that result back. Otherwise, if the primitive was a Number, push that Number back unchanged.
		\item[if Pop a primitive and then two Objects from Stack B. If the primitive is an empty Object, the null Identifier, or the Number 0, explode the first Object popped. Else, the second.
		\item[obj!] Pop a primitive from Stack B. If that primitive not an Object, push an Object representing each individual character in the primitive’s representation to the stack (as with String), else an empty Object.
		\item[@] Pop a primitive from Stack B. If that primitive was an Identifier, push a new Object containing it onto the stack. Otherwise, push an empty Object.
	\end{description}
	\section{Standard Library}
\end{document}
